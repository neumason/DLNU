\begin{resume}

  \resumeitem{个人简历}

%  1980年2月3日出生于吉林省辽源市;
%
%  2002年5月考入东北大学计算机软件与理论专业,2005 年5月获得工学硕士学位;
%  
%  2005年6月至今,大连民族大学计算机科学与工程学院,教师;
%  
%  2010年9月考入东北大学控制理论与控制工程专业在职攻读博士学位。


  \resumeitem{第一作者发表/录用的学术论文} % 发表的和录用的合在一起

  \begin{enumerate}[{[}1{]}]  

\item 第一作者. Facial semantic representation for ethnical Chinese minorities based on geometric similarity[J]. International Journal of Machine Learning \& Cybernetics, 2017(Pt 3):1-21.(SCI刊源,在线发表)
  
%\item WANG Cun-Rui, ZHANG Qing-Ling, DUAN Xiao-Dong, WANG Yuan-Gang, LI Ze-Dong. Research of Face Ethnic Features from Manifold Structure. Acta Automatica Sinica, 2018, 44(1): 140-159.
  
\item 第一作者. Multi-ethnical Chinese Facial Characterization and Analysis   Multimedia Tools and Applications.  (DOI: 10.1007/s11042-018-6018-1)(SCI刊源,已录用)

\item 第一作者. Facial Feature Discovery for Ethnicity Recognition[J]. WIREs Data Mining and Knowledge Discovery (DMKD-00355) (SCI刊源,已录用)

\item 第一作者.  基于流形结构的人脸民族特征研究[J]. 自动化学报, 2018, 44(1). (EI检索期刊,已刊发)

\item 第一作者.  基于尺度聚类仿射过滤的图像拼接算法[J]. 东北大学学报(自然科学版), 2011, 32(7):917-921. EI检索期刊,已刊发)

%\item 第一作者.   A network weight-based Multi-Community discovery Algorithm[C]. Third International Workshop on Advanced Computational Intelligence. IEEE, 2010:570 - 574.


\item 第一作者. 生物地理学优化算法综述[J]. 计算机科学, 2010, 37(7):34-38.
%\item 第一作者. 一种基于SIFT的网格仿射拼接算法[J]. 计算机工程与应用, 2011, 47(36):187-189.
    

\end{enumerate}
\resumeitem{合作发表的学术论文}
\begin{enumerate}[{[}1{]}]
\item  第四作者. Multi-ethnic facial features extraction based on axiomatic fuzzy set theory[J], Neurocomputing, 2017, 242, 161-177. (SCI检索号:000399859500014)

\item 第四作者. New semantic descriptor construction for facial expression[J], Multimedia Tools and Applications,DOI: 10.1007/s11042-017-4818-3.(SCI, 已录用)

\item 第三作者. 基于AFS的多民族人脸语义描述与挖掘方法研究[J]. 计算机学报, 2016, 39(7):1435-1449.

\item 第五作者. AFSNN: a classification algorithm using axiomatic fuzzy sets and neural networks[J]. \textbf{IEEE Transactions on Fuzzy Systems}, 2018, 10.1109/TFUZZ.2017.2788875.\textbf{(SCI,已发表})

\item 第四作者. A spectral clustering method with semantic description based on axiomatic fuzzy sets theory[J]. \textbf{Applied Soft Computing}, 2018, 64:59-74. \textbf{(SCI, 已发表)}

%\item Yuangang Wang, Xiaodong Duan, Xiaodong Liu, Cunrui Wang, Zedong Li. Semantic description method for face features of larger Chinese ethnic groups based on improved WM method[J]. \textbf{Neurocomputing}, 2016, 175: 515-528. \textbf{(SCI,已发表)}

\item 第三作者.  Comparison and Fusion of Multiple Types of Features for Image-Based Facial Beauty Prediction[C]. Chinese Conference on Biometric Recognition. Springer, Cham, 2017:23-30.

\item 第四作者.  Semantic description method for face features of larger Chinese ethnic groups based on improved WM method[J]. Neurocomputing, 2016, 175(PA):515-528.

\item 第四作者.  A Computational Other-Race-Effect Analysis for 3D Facial Expression Recognition[M]. Biometric Recognition. Springer International Publishing, 2016:483-493.

\item 第三作者. Research on Facial Features of Major Chinese Nationalities[J]. Applied Mechanics \& Materials, 2014, 687-691:837-840.


	\end{enumerate}

\resumeitem{研究成果} % 有就写,没有就删除
  \begin{enumerate}[{[}1{]}]
  \item 人脸信息采集系统: 中国,2013SR008567. (软件著作权.)
  \item 人脸2D-3D模型构建系统: 中国,2013SR008932. (软件著作权.)
  \end{enumerate}
  
  
  {\large 参加的科研项目}
\begin{enumerate}
\item 国家自然科学基金项目“基于族群谱系特征的多民族表情理解研”(61370146),主要参与者。

\item 国家自然科学基金项目``眼睛与面部动作在自发表情中的协同机制研究及应用”(61672132)“,主要参与者。
\end{enumerate}

\newpage
\quad

\end{resume}
