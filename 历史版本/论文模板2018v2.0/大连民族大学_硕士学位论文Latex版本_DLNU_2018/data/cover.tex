
%%% Local Variables:
%%% mode: latex
%%% TeX-master: t
%%% End:

\ctitlehead{硕\;士\;学\;位\;论\;文} %出发需要, 否则不要更改 ctitlehead
% 再这里往下开始更改
\ctitle{人脸疲劳检测算法研究及其在嵌入式平台的实现}
\id{TP311.1}

\studentID{20121101109}
\secretlevel{公开} 
\secretyear{2100}
\datesubmit{2018年10月} %提交时间
\datedefend{2018年12月26日} %答辩时间

% 根据自己的情况选,不用这样复杂
\makeatletter
\ifthu@bachelor\relax\else
  \ifthu@doctor
    \cdegree{工学博士}
  \else
    \ifthu@master
      \cdegree{工学硕士}
    \fi
  \fi
\fi
\makeatother


\cdepartment[计算机科学与工程学院]{计算机科学与工程学院}
\cmajor{计算机技术}
\cauthor{XXX} 
%\cauthor{}
\date{{\the\year}年{\the\month}月}%论文答辩月份
\csupervisor{XXX\;副教授\;XXX学院}
%\csupervisor{}
% 如果没有副指导老师或者联合指导老师,把下面两行相应的删除即可。
\cassosupervisor{XXX\;XXX\;XXX公司或学院}
%\cassosupervisor{}
%\ccosupervisor{某某某教授}
% 日期自动生成,如果你要自己写就改这个cdate
\cdate{\CJKdigits{\the\year}年\CJKnumber{\the\month}月}

% 博士后部分
% \cfirstdiscipline{计算机科学与技术}
% \cseconddiscipline{系统结构}
% \postdoctordate{2009年7月——2011年7月}

\etitle{The Study of Face Fatigue Detection Algorithms and Its Implementation on Embedded Platformwang} 
% \edegree{Doctor of Science} 
\edegree{Doctor of Engineering}
\emajor{Computer Technology} 
\eauthor{XXXX} 
\esupervisor{Associate Professor XXXX} 
\eassosupervisor{Professor XXXXXXXX} 
% \eauthor{} 
% \esupervisor{} 
% \eassosupervisor{} 
% 这个日期也会自动生成,你要改么?
% \edate{December, 2005}

% 定义中英文摘要和关键字

\begin{cabstract}
疲劳是人类正常的生理现象,面部的眼睛疲惫、频繁的打哈欠、精力不集中均是疲劳的表现形式。而由疲劳造成的危害也有很多,疲劳驾驶、长时间使用手机造成的视线模糊、心情烦躁等等。因此,人脸疲劳检测成了一项重要的研究内容。此外,人们使用嵌入式移动设备的时间不断增加,如何避免过度使用移动设备或者预防游戏视觉疲劳都具有重要的应用价值。

本文拟构建基于嵌入式平台的人脸疲劳检测方法及应用系统。首先利用人脸检测和人脸特征点定位对人脸进行标定,然后构建相应方法对人脸进行嘴部和眼部的疲劳分析。本文提出了一种适用于嵌入式平台的PERCLOS-A疲劳检测方法,并通过实验验证其有效性,同时将其移植到嵌入式平台对人脸进行疲劳检测。检测到人脸的眼部区域和嘴部区域,并对该区域实现特征点提取与分析,为后续的实验研究做好数据铺垫。

本文工作包括2个方面:

1)通过研究眼部、嘴部以及嘴眼协调参数变化对人脸疲劳进行分析。通过查阅和实验,发现传统的PERCLOS(percentage of eyelid closure over the pupil over time)疲劳检测算法和哈欠特征频率计算方法不适用于嵌入式平台,本文通过大量实验和论述对参数进行改进,进而提出了一种适用于嵌入式平台的疲劳检测参数的计算方法,并且通过实验对比了不同方法的优劣。

2)本文将人脸疲劳检测方法植入嵌入式ARM平台,建立了人脸识别的疲劳检测系统,系统中采用模块化的设计思路,将眼部和嘴部的特征识别功能自由地组合在一起,然后实现一个基于人脸识别的疲劳检测系统,并探讨针对特定人群疲劳检测系统的可能性和可行方案,最终实现对人脸的疲劳检测。检测条件限制减少并能脱离实验室环境,可以使用Android设备进行检测,应用环境简单方便。

本文通过综合以上提出的基于人脸识别的疲劳检测框架和基于PERCLOS算法改进的方案,并结合嵌入式平台技术分别以拍照和视频检测两种形式版本的疲劳检测系统原型,为进一步提高疲劳检测准确率确立了基础。




\end{cabstract}

\ckeywords{疲劳检测;人脸检测;嵌入式平台}

\begin{eabstract} 
\indent
Fatigue is the normal physiological phenomenon of human, especially facial features reflected fatigue are obvious such as facial tired eyes, frequent yawn, inattention. As a result, there are a lot of harms caused by fatigue such as fatigue driving, blurred vision caused by using the phone for long time, mood irritability. In addition, as increasing time which people use embedded mobile devices, how to detect game fatigue caused by overusing has significant application value. Therefore, face fatigue detection becomes an important research central issue.
Taking above-mentioned factors, a novel embedded platform-based face fatigue detection application system is constructed in this paper. First, locate face through face detection and face feature location, then analyze the fatigue of mouth and eye via corresponding method. As the core of system, a percentage of eyelid closure over the pupil over time (PERCLOS) fatigue detection parameter method is proposed, where the area of eye and mouth is located for extracting and analyzing face feature points. This method is also transplanted to embedding platform for detecting face fatigue, and experiment result show that the PERCLOS-A fatigue detection parameter method still has high accuracy in case of real dataset. As previous work, experiment data also is used for further study. 

The main contributions of this work are:

1) An improved PERCLOS method is proposed for face fatigue detection, this is due to the traditional methods of eyelid closure over the pupil over time fatigue detection and yaw characteristic frequency calculation are not suitable for the embedded platform. In our method, parameters in traditional PERCLOS are modified through analyzing change of coordination parameters of the eyes and mouth. In addition, this method is compared with other methods by transplanting to embedding platform.
 
2) In this thesis, the face fatigue detection system is built through transplanting this method into the embedded platform ARM. The modular design idea is used to build the system. In details, the feature recognition function of eyes and mouth is freely combined for realizing a fatigue detection system based on face recognition. 

In this thesis, face recognition-based fatigue detection framework and the improved PERCLOS algorithm are combined with the embedded platform technology to take the prototype of two kinds of fatigue detection system, such as camera and video detection, in order to further improve the fatigue detection rate established the basis.


\end{eabstract}

\ekeywords{Intelligent control; Machine vision; Facial ethnic features; Feature Extraction}
